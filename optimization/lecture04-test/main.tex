\documentclass{article}

\usepackage{amsmath}
\usepackage{amssymb}
\usepackage{graphicx}
\usepackage{hyperref}
\hypersetup{
  colorlinks=true,
  linkcolor=blue, 
}

\newcommand{\E}{\mathbb{E}}
\newcommand{\R}{\mathbb{R}}
\newcommand{\N}{\mathbb{N}}
\newcommand{\1}{\mathbf{1}}
\renewcommand{\P}{\mathbb{P}}
\renewcommand{\L}{\mathcal{L}}
\newcommand{\ie}{\textit{i.e. }}
\newcommand{\eg}{\textit{e.g. }}

\title{Optimization - Minitest 2}
\author{Joris LIMONIER}
\begin{document}
\maketitle
% \tableofcontents
      
\section{Question 1}
\paragraph{What is a surrogate loss?}
\begin{itemize}
  \item A loss that we use instead of the natural loss.
  \item It is greater than the surrogate loss.
  \item It is convex in the number of parameters.
\end{itemize}

\section{Question 2}
\paragraph{Why do we use it in Machine Learning?}
Because convex optimization problems are easier to solve.


\end{document}