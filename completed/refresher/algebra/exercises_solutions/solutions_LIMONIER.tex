\documentclass{article}

\usepackage{amsmath}
\usepackage{amssymb}

\renewcommand{\P}{\mathbb{P}}
\newcommand{\R}{\mathbb{R}}
\newcommand{\Z}{\mathbb{Z}}
\newcommand{\N}{\mathbb{N}}

\title{Algebra refreshers\\Exercises}
\author{Joris LIMONIER}
\begin{document}
\maketitle

\tableofcontents

\setcounter{section}{4}
\section{Matrices, systems of linear equations and determinants}
\subsection{Matrix algebra}
\subsubsection{}
\begin{align*}
             & A
    = \begin{bmatrix}
        1  & 2 & 4  \\
        -3 & 0 & -1 \\
        2  & 1 & 2
    \end{bmatrix} \\
    \implies & 3A
    = \begin{bmatrix}
        3  & 6 & 12 \\
        -9 & 0 & -3 \\
        6  & 3 & 6
    \end{bmatrix} \\
\end{align*}

\subsubsection{}
\begin{align*}
    3A - B & =
    \begin{bmatrix}
        3  & 6 & 12 \\
        -9 & 0 & -3 \\
        6  & 3 & 6
    \end{bmatrix} -
    \begin{bmatrix}
        2  & 0  & 0 \\
        1  & -4 & 3 \\
        -1 & 3  & 2
    \end{bmatrix} \\
           & =
    \begin{bmatrix}
        1   & 6 & 12 \\
        -10 & 4 & -6 \\
        7   & 0 & 4
    \end{bmatrix}
\end{align*}

\subsubsection{}
\begin{align*}
    AB & =
    \begin{bmatrix}
        1  & 2 & 4  \\
        -3 & 0 & -1 \\
        2  & 1 & 2
    \end{bmatrix}
    \begin{bmatrix}
        2  & 0  & 0 \\
        1  & -4 & 3 \\
        -1 & 3  & 2
    \end{bmatrix}  \\
       & =
    \begin{bmatrix}
        2  & 0 & 0 \\
        -6 & 0 & 0 \\
        4  & 0 & 0
    \end{bmatrix} +
    \begin{bmatrix}
        2 & -8 & 6 \\
        0 & 0  & 0 \\
        1 & -4 & 3
    \end{bmatrix} +
    \begin{bmatrix}
        -4 & 12 & 8  \\
        1  & -3 & -2 \\
        -2 & 6  & 4
    \end{bmatrix} \\
       & =
    \begin{bmatrix}
        0  & 4  & 14 \\
        -5 & -3 & -2 \\
        3  & 2  & 7
    \end{bmatrix} \\
\end{align*}


\subsubsection{}
\begin{align*}
    BA & =
    \begin{bmatrix}
        2  & 0  & 0 \\
        1  & -4 & 3 \\
        -1 & 3  & 2
    \end{bmatrix}
    \begin{bmatrix}
        1  & 2 & 4  \\
        -3 & 0 & -1 \\
        2  & 1 & 2
    \end{bmatrix} \\
       & =
    \begin{bmatrix}
        2  & 4  & 8  \\
        1  & 2  & 4  \\
        -1 & -2 & -4
    \end{bmatrix} +
    \begin{bmatrix}
        0  & 0 & 0  \\
        12 & 0 & 4  \\
        -9 & 0 & -3
    \end{bmatrix} +
    \begin{bmatrix}
        0 & 0 & 0 \\
        6 & 3 & 6 \\
        4 & 2 & 4
    \end{bmatrix} \\
       & =
    \begin{bmatrix}
        2  & 4 & 8  \\
        19 & 5 & 14 \\
        -6 & 0 & -3
    \end{bmatrix}
\end{align*}

\subsubsection{}

\begin{align*}
    C(3A - 2B)
     & =
    C
    \left( 3
    \begin{bmatrix}
        1  & 2 & 4  \\
        -3 & 0 & -1 \\
        2  & 1 & 2
    \end{bmatrix} - 2
    \begin{bmatrix}
        2  & 0  & 0 \\
        1  & -4 & 3 \\
        -1 & 3  & 2
    \end{bmatrix}
    \right)                    \\
     & =
    C
    \left(
    \begin{bmatrix}
        3  & 6 & 12 \\
        -9 & 0 & -3 \\
        6  & 3 & 6
    \end{bmatrix} -
    \begin{bmatrix}
        4  & 0  & 0 \\
        2  & -8 & 6 \\
        -2 & 6  & 4
    \end{bmatrix}
    \right)                    \\
     & =
    \begin{bmatrix}
        2  & 1 & 0  \\
        1  & 0 & 3  \\
        -1 & 0 & 2  \\
        4  & 5 & -1
    \end{bmatrix}
    \begin{bmatrix}
        -1  & 6  & 12 \\
        -11 & 8  & -9 \\
        8   & -3 & 2
    \end{bmatrix} \\
     & =
    \begin{bmatrix}
        2  & 1 & 0  \\
        1  & 0 & 3  \\
        -1 & 0 & 2  \\
        4  & 5 & -1
    \end{bmatrix}
    \begin{bmatrix}
        -1  & 6  & 12 \\
        -11 & 8  & -9 \\
        8   & -3 & 2
    \end{bmatrix} \\
     & =
    \begin{bmatrix}
        -2 & 12 & 24  \\
        -1 & 6  & 12  \\
        1  & -6 & -12 \\
        -4 & 24 & 48
    \end{bmatrix} +
    \begin{bmatrix}
        -11 & 8  & -9  \\
        0   & 0  & 0   \\
        0   & 0  & 0   \\
        -55 & 40 & -45
    \end{bmatrix} +
    \begin{bmatrix}
        0  & 0  & 0  \\
        24 & -9 & 6  \\
        16 & -6 & 4  \\
        -8 & 3  & -2
    \end{bmatrix} \\
     & =
    \begin{bmatrix}
        -13 & 20  & 15 \\
        23  & -3  & 18 \\
        17  & -12 & -8 \\
        -67 & 67  & 1
    \end{bmatrix} \\
\end{align*}

\subsection{}
\begin{align*}
    \begin{bmatrix}
        1 & 2 & -3
    \end{bmatrix}
    \begin{bmatrix}
        2 \\
        1 \\
        5
    \end{bmatrix}
    = 2 + 2 - 15 = -11
\end{align*}
\begin{align*}
    \begin{bmatrix}
        2 \\
        1 \\
        5
    \end{bmatrix}
    \begin{bmatrix}
        1 & 2 & -3
    \end{bmatrix}
    =
    \begin{bmatrix}
        2 & 4  & -6  \\
        1 & 2  & -3  \\
        5 & 10 & -15
    \end{bmatrix}
\end{align*}

\subsection{}
Let \(m, n, p \in \N\) such that \(A \in \mathcal{M}_{m, n} (\R)\) and \(B \in \mathcal{M}_{n, p} (\R)\), then \(AB \in \mathcal{M}_{m,p} (\R)\). Since \(AB\) is squared, we have that \(m = p\). Thus \(B \in \mathcal{M}_{n,p} = \mathcal{M}_{n,m}\) so \(BA \in \mathcal{M}_{n,n}\) is well defined.

\subsection{}
\(c_{13} = 1 \times 0 + 2 \times 3 + 1 \times 2 = 8\) \\
\(c_{22} = -3 \times 0 + 0 \times -4 + (-1) \times 3 = -3\)

\subsection{}
\subsection{}
(a) and (b)

\begin{align*}
    A   & =
    \begin{bmatrix}
        2 & 0  & 0 \\
        0 & -1 & 0 \\
        0 & 0  & 3
    \end{bmatrix} \\
    A^n & =
    \begin{bmatrix}
        2^n & 0      & 0   \\
        0   & (-1)^n & 0   \\
        0   & 0      & 3^n
    \end{bmatrix} \\
\end{align*}

(c)
Initialization: let \(k=0\). \(D^0 = I \implies \forall 1 \leq i \leq r, D_{ii} = 1 = \lambda^0\)

\subsection{}
\[
    A =
    \begin{bmatrix}
        1  & -2 \\
        -2 & 3
    \end{bmatrix}
    \qquad
    B =
    \begin{bmatrix}
        -2 & 1 \\
        1  & 1
    \end{bmatrix}
\]

\begin{align*}
    & AB =
    \begin{bmatrix}
        1  & -2 \\
        -2 & 3
    \end{bmatrix}
    \begin{bmatrix}
        -2 & 1 \\
        1  & 1
    \end{bmatrix} \\
    \implies & AB =
    \begin{bmatrix}
        -2 & 1 \\
        4 & -2
    \end{bmatrix} +
    \begin{bmatrix}
        -2 & 3 \\
        -2 & 3 
    \end{bmatrix} \\
    \implies &
    AB =
    \begin{bmatrix}
        -4 & 4 \\
        2 & 1
    \end{bmatrix} \\
    \implies &
    (AB)^t =
    \begin{bmatrix}
        -4 & 2 \\
        4 & 1
    \end{bmatrix}
\end{align*}

\section{Exercises from lecture notes}
\subsection{Exercise p. 20.}
\subsubsection{Prove that \(v - P_{U}(v) \in U^{\perp}\)}

\begin{align*}
    v - P_U (v)
    &= \underbrace{u}_{=P_U (v)} + u^{\perp} - P_U(v) \\
    &= P_U (v) + u^{\perp} - P_U(v) \\
    &= u^{\perp} \\
    &\in U^\perp
\end{align*}

\subsubsection{Exercise}
Prove that, given $v \in V$ and $u \in U \subset V$, then
\[
P_{U}(v)=\arg \min _{u \in U}\|v-u\|^{2}
\]
In other words, prove that the orthogonal projection of $v$ on $U$ is the nearest point of $U$ to $v$ (Hint: use that $\left.v-u=v-P_{U}(v)+P_{U}(v)-u\right)$.

\section{Extra exercises}
\subsection{Exercise 1}
\begin{align*}
    \frac{\partial}{\partial x} \left[ (x^2 + y^2)^{1/2} \right]
    &= \frac{x}{(x^2 + y^2)^{1/2}}
\end{align*}

\subsection{Exercise 2}
\begin{align*}
    \frac{\partial}{\partial x} \left[ \log (x^2 + y^2) \right]
    &= \frac{2x}{x^2 + y^2}
\end{align*}
\end{document}