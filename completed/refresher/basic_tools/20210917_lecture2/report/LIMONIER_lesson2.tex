\documentclass{article}

\usepackage{amsmath}
\usepackage{amssymb}

\renewcommand{\P}{\mathbb{P}}

\title{Basic tools refreshers:\\Homework for lecture 2}
\author{Joris LIMONIER}
\begin{document}
\maketitle


\begin{enumerate}
    \item mkdir lesson\_1
    \item mv tic lesson\_1/tic
    \item mkdir lesson\_2 \\ cd lesson\_2
    \item touch file1.1.1 \\
    echo -e ``Hello\(\backslash\)nHello\(\backslash\)nHello" \(>>\) file 1.1.1 \\
    echo -e ``world\(\backslash\)nworld" \(>>\) file 1.1.1 \\
    echo -e ``*" \(>>\) file 1.1.1
    \item cat file1.1.1 \\
    less file1.1.1
    \item Prints all subdirectories from root.
    \item tree /etc \(>\) tree\_etc
    \item 
    \item 
    \item wget http://download.geonames.org/export/zip/FR.zip \\
    unzip FR.zip -d ./ \\
    rm FR.zip
    \item less FR.txt \\
    This file is a table.
    \item grep "01000" FR.txt \\
    The cities are Bourg-en-Bresse and Saint-Denis-lès-Bourg.
    \item (Not to be done)
    \item mv FR.txt french\_zipcodes.txt
    \item ls french\_zipcodes.txt -l \\
    The user and the group have read and write rights, whereas the other users only have read right.
    \item
    \begin{enumerate}
        \item chmod u+rwx french\_zipcodes.txt
        \item chmod u-x french\_zipcodes.txt
        \item chmod u-w french\_zipcodes.txt
    \end{enumerate}
    \item chmod u+rwx french\_zipcodes.txt \\
    chmod u-w french\_zipcodes.txt
    \item Because the rights is "- r - - r w - r - -", group users already have read and write (\textit{i.e.} edit). Now to prevent them from deleting this file, we run the following commands:\\
    cd .. \\
    ls -l (to check the current rights for lesson\_2, which returns drwxrwxr-x) \\
    chmod g-w lesson\_2/ (and now the rights of lesson\_2 are drwxr-xr-x) \\
    In order to allow group users to delete french\_zipcodes.txt, we would have to run chmod g+w lesson\_2/.
    \item HELLO=``Hello, World !" \\
    echo \$HELLO
    \item for entry in ../lesson\_1/* \\
    do \\
    \text{ } echo ``\$entry" \\
    done \(>>\) LONG\_FILES\_LIST
    \item touch user\_reply \\
    read reply \\
    echo \$reply > user\_reply
    \item touch hello.sh \\
    nano hello.sh \\
    read user\_name \\
    echo "Hello \$user\_name !" \\
    ctrl+X \\
    \([\) y \(]\) \\
    \([\) Enter \(]\) \\
    chmod a+x hello.sh \\
    ./hello.sh \\
    Joris \\
    (prints Hello Joris !)
    \item Let us first create a file in which to print "Error message": \\
    touch err\_mess \\
    Now we create a shell file and make it executable: \\
    touch q23.sh \\
    chmod a+x q23.sh \\
    nano q23.sh \\
    echo "Normal message" \\
    echo "Error message" >> \$1 \\
    Ctrl + X \\
    y \\
    Enter \\
    ./q23.sh err\_mess \\
    As expected, "Normal message" is printed in the terminal and "Error message" is written to the file err\_mess.
    \item mv \$1 \$2 \\
    mv \$2 "\$3/\$2" \\
    echo "Nb of arguments is \$\#" \\
    echo "Arguments from input are \$1 \$2 \$3"
    \item for i in \$(seq 50 99) \\
    do \\
    \mbox{  } touch file\$i \\
    done
    \item rm file[5-9][0-9]
    \item \begin{enumerate}
      \item for i in \$(seq 1 \$1) \\
      do \\
      \text{ } touch \$3"/\$2\$i" \\
      done
      \item for i in \$(seq 1 \$1) \\
      do \\
      \text{ } mv "\$2/\$3\$i" "\$2/\$3\$i.txt" \\
      done
      \item for name in \$(ls \*.txt) \\
      do \\
      \text{ } mv "\$1\$name" "\$2\$name" \\
      done
    \end{enumerate}
\end{enumerate}

\end{document}