\documentclass[12pt]{article}
\usepackage[utf8]{inputenc}

% Importing settings from setup.sty
\usepackage{setup}
\usepackage{booktabs}
\usepackage{multicol}
\usepackage{multirow}
\usepackage{glossaries}
% \makenoidxglossaries
% \newcommand{\prox}{\operatorname{prox}}


% \pagenumbering{roman}
\begin{document}

% Inserting title page
\import{./}{title}

\pagenumbering{gobble}
\tableofcontents
% \listoffigures
% \listoftables



\newgeometry{
  left=25mm,
  right=25mm,
  top=25mm,
  bottom=25mm}
\pagenumbering{arabic}
 
% Evaluation: Write a brief report in which you provide your own view of the challenge of "Learning Without Data Collections". The report can be based on my book entitled "DEEP LEARNING TO SEE - Towards New Foundations of Computer Vision" and on both my two lectures. In particular you could organized you report as follows:

% 1. Brief discussion on actual contexts in which  the view of  "Learning Without Data Collections" can take place.
% 2. Critical comments on the notion of motion invariance. In particular, provide a perspective on strengths and weakness in the process of feature extraction
% 3. Find one-two articles that you consider related to the covered topic and underline the similarities.

% Frogs' eyes can't see what doesn't move.
% AI are really good (performance), but really bad (computational cost)

\section{Introduction}
``Learning without data collection'' is a provocative way to describe the process of learning from a single, or very few example. Humans however, are able to learn in such a way. Algorithms on the other hand, especially deep learning ones, are very data-hungry even for simple tasks such as recognising a cat from a dog.

\section{Context of Learning without Data Collection}
Learning without data collection can be applied to many fields within machine learning. One notable example is the field of computer vision. When humans see a object moving, they are almost imidiately able to recognise it. We do not require hundreds of thousands of examples in order to identify it. One iteresting thing however is that, when we see a paused video, on a phone or a computer screen for instance, we are sometimes incapable of recognising objects. Think of a paused video about a jaguar in the jungle. The jaguar may be partially, or almost completely hidden. In this case, it may be difficult to identify it. When we turn the video on, it may still be challenging to see the jaguar if he is immobile. However, as soon as the jaguar starts moving, we are able to identify it. This is because of motion invariance \cite{gori2022}, which informs us of the consistency of an object (the jaguar in this case) through time. Indeed, we know that the paws, the ears, the tail and all other body parts of the jaguar will not start splitting up or drastically changing shape as time goes. We also do not need to classify each pixel of the image we see in order to identify the boundaries of the body of the jaguar.
\\
Furthermore, we do not even need labels in order to understand the consistency of one class. In that sense, Humans are good unsupervised machines.


\section{Critical comments on the notion of motion invariance}
\section{Related articles}
% \clearpage
% \printnoidxglossaries

\begin{thebibliography}{99}
  \bibitem{gori2022} Betti, A., Gori, M. and Melacci, S., 2022. Deep Learning to See: Towards New Foundations of Computer Vision. arXiv preprint arXiv:2206.15351.  
\end{thebibliography}
\end{document}

