\documentclass[12pt]{article}
\usepackage[utf8]{inputenc}

% Importing settings from setup.sty
\usepackage{setup}
\usepackage{booktabs}
\usepackage{multicol}
\usepackage{multirow}
\usepackage{glossaries}
% \makenoidxglossaries
% \newcommand{\prox}{\operatorname{prox}}


% \pagenumbering{roman}
\begin{document}

% Inserting title page
\import{./}{title}

\pagenumbering{gobble}
\tableofcontents
% \listoffigures
% \listoftables



\newgeometry{
  left=25mm,
  right=25mm,
  top=25mm,
  bottom=25mm}
\pagenumbering{arabic}
 
% Evaluation: Write a brief report in which you provide your own view of the challenge of "Learning Without Data Collections". The report can be based on my book entitled "DEEP LEARNING TO SEE - Towards New Foundations of Computer Vision" and on both my two lectures. In particular you could organized you report as follows:

% 1. Brief discussion on actual contexts in which  the view of  "Learning Without Data Collections" can take place.
% 2. Critical comments on the notion of motion invariance. In particular, provide a perspective on strengths and weakness in the process of feature extraction
% 3. Find one-two articles that you consider related to the covered topic and underline the similarities.

% Frogs' eyes can't see what doesn't move.
% AI are really good (performance), but really bad (computational cost)

\section{Introduction}
``Learning without data collection'' is a provocative way to describe the process of learning from a single, or very few example. Humans however, are able to learn in such a way. Algorithms on the other hand, especially deep learning ones, are very data-hungry even for simple tasks such as recognising a cat from a dog.

\section{Context of Learning without Data Collection}
Learning without data collection can be applied to many fields within machine learning. One notable example is the field of computer vision. When a human 


\section{Critical comments on the notion of motion invariance}
\section{Related articles}
% \clearpage
% \printnoidxglossaries

\begin{thebibliography}{99}
  \bibitem{gori2022} Betti, A., Gori, M. and Melacci, S., 2022. Deep Learning to See: Towards New Foundations of Computer Vision. arXiv preprint arXiv:2206.15351.  
\end{thebibliography}
\end{document}