\documentclass{article}

\usepackage{amsmath}
\usepackage{amssymb}
\usepackage{amsfonts}
\usepackage{graphicx}
\usepackage{hyperref}
\hypersetup{
  colorlinks=true,
  linkcolor=blue, 
}

\newcommand{\1}{\mathbf{1}}
\newcommand{\E}{\mathbb{E}}
\newcommand{\R}{\mathbb{R}}
\newcommand{\Rcal}{\mathcal{R}}
\renewcommand{\P}{\mathbb{P}}
\renewcommand{\L}{\mathcal{L}}
\newcommand{\thetahat}{\hat\theta}
\newcommand{\ie}{\textit{i.e. }}
\newcommand{\var}{\operatorname{Var}}


\title{Statistical learning theory}
\author{Joris LIMONIER}
\begin{document}
\maketitle

\tableofcontents

\section{Proposed exercise lecture 02}
\subsection{Home Exercise}
\paragraph{Demonstrate that the MAP (Maximum A Posteriori) classification rule is quadratic with respect to $x^*$.}

We have the following:
\begin{equation}
  \hat{y}^* = \arg \max_{k = 1, \ldots, k} \P \left( Y=k | X = x^* \right)
\end{equation}
We compute the probability:
\begin{align*}
  \arg \max_{k = 1, \ldots, k} \P \left( Y=k | X = x^* \right)
   & \underbrace{=}_{\text{Bayes}} \frac{\P(Y=k) \P(X=x^* | Y=k)}{\P(X=x^*)}        \\
   & \propto \P(Y=k) \P(X=x^* | Y=k)                                                \\
   & \propto \hat{\Pi}_k \mathcal{N}\left( x^*, \hat{\mu}_k, \hat{\Sigma}_k \right) \\
\end{align*}

\begin{align*}
  \hat{y}^*
   & = \arg \max_{k = 1, \ldots, k} \hat{\Pi}_k \mathcal{N}\left( x^*, \hat{\mu}_k, \hat{\Sigma}_k \right)                                                                                                                \\
   & = \arg \max_{k = 1, \ldots, k} \log \left[ \hat{\Pi}_k \right] + \log \left[\mathcal{N} \left( x^*, \hat{\mu}_k, \hat{\Sigma}_k \right) \right]                                                                      \\
   & = \arg \max_{k = 1, \ldots, k} \log \left[ \hat{\Pi}_k \right] + \log \left[\frac{1}{(2 \pi)^{N / 2}|\Sigma|^{1 / 2}} e^{-\frac{1}{2}(x^*-\mu)^{\top} \Sigma^{-1}(x^*-\mu)} \right]                                  \\
   & = \arg \max_{k = 1, \ldots, k} \underbrace{\log \left[ \hat{\Pi}_k \right]}_{\text{Constant w.r.t. } x^*} + \underbrace{\left[ -\frac{1}{2}(x^*-\mu)^{\top} \Sigma^{-1}(x^*-\mu) \right]}_{\text{Quadratic in } x^*} \\
   & - \underbrace{\log \left[(2 \pi)^{N / 2}|\Sigma|^{1 / 2} \right]}_{\text{Constant w.r.t. } x^*}                                                                                                                      \\
\end{align*}
The second term is Quadratic in $x^*$, the others are constant w.r.t. $x^*$, which completes the proof.






\end{document}