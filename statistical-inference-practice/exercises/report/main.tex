\documentclass{article}

\usepackage{amsmath}
\usepackage{amssymb}
\usepackage{graphicx}
\usepackage{hyperref}
\hypersetup{
  colorlinks=true,
  linkcolor=blue, 
}

\newcommand{\E}{\mathbb{E}}
\newcommand{\R}{\mathbb{R}}
\renewcommand{\P}{\mathbb{P}}
\renewcommand{\L}{\mathcal{L}}
\newcommand{\thetahat}{\hat\theta}
\newcommand{\ie}{\textit{i.e. }}

\title{Statistical inference practice}
\author{Joris LIMONIER}
\begin{document}
\maketitle

\tableofcontents

\section{Homework for October 8, 2021}
\subsection{Chapter 6 - Exercise 2}

Let \(X_1, \ldots, X_n\) be i.i.d. random variables with distribution \(\mathcal{U}(0, \theta)\). \\
Let \(\hat\theta_n = \max(X_1, \ldots, X_n)\).

\paragraph{What is the bias of \(\hat\theta_n\) ?\\}
\(\hat\theta_n\) is unbiased if \(\E[\hat\theta_n] = \theta\). \\
We compute \(F_{\thetahat_n}\), the CDF of \(\hat\theta_n\):
\begin{align*}
    F_{\thetahat_n}(x)
     & = \P(\thetahat \leq x)                                                       \\
     & = \P(X_1 \leq x, \ldots, X_n \leq x)                                         \\
     & = \P(X_1 \leq x) \ldots \P(X_n \leq x) & \textit{(independence)}             \\
     & = \left[\P(X_1 \leq x)\right]^n        & \textit{(identity of distribution)} \\
     & = \begin{cases}
        0                                 & x < 0                \\
        \left[ \frac{x}{\theta} \right]^n & 0 \leq x \leq \theta \\
        1                                 & x > \theta
    \end{cases}
\end{align*}
Then \(f_{\thetahat_n}\), the PDF of \(\thetahat_n\) is given by:
\begin{align*}
    f_{\thetahat_n}(x)
     & = \frac{d}{dx} F_{\thetahat_n}(x) \\
     & =
    \begin{cases}
        \frac{n x^{n-1}}{\theta^n} & x \in [0, \theta] \\
        0                          & \text{otherwise}
    \end{cases}
\end{align*}
Hence the expected value is given by:
\begin{align*}
    \E[\hat\theta_n]
     & = \int_{-\infty}^{+\infty} t f(t) dt                         \\
     & = \frac{n}{\theta^n} \int_{0}^{\theta} t^n dt                \\
     & = \frac{n}{\theta^n (n+1)} \left[t^{n+1}\right]_{0}^{\theta} \\
     & = \frac{n}{\theta^n (n+1)} \left[ \theta^{n+1} - 0 \right]   \\
     & = \frac{n \theta}{n+1}                                       \\
\end{align*}
Since \(\E[\hat\theta_n] = n \theta \neq \theta\), we have that \(\thetahat_n\) is not an unbiased estimator for \(\theta\). However, \(\E[\thetahat_n] \xrightarrow[n \to \infty]{} \theta\), therefore \(\thetahat_n\) is asymptotically unbiased.


\paragraph{What is \(SE\), the standard error of \(\hat\theta_n\) ?\\}

\begin{align}
    SE
    \nonumber
     & = SE(\thetahat_n)                              \\
    \nonumber
     & = \sqrt{Var(\thetahat_n)}                      \\
    \label{eqn: ex2 SE}
     & = \sqrt{\E[\thetahat_n^2] - \E[\thetahat_n]^2}
\end{align}
We need to find \(\E[\thetahat_n^2]\).
\begin{align*}
    \E[\thetahat_n^2]
     & := \int_{-\infty}^{+\infty} t^2 f(t) dt                      \\
     & = \frac{n}{\theta^n} \int_{0}^{\theta} t^{n+1} dt            \\
     & = \frac{n}{\theta^n (n+2)} \left[t^{n+2}\right]_{0}^{\theta} \\
     & = \frac{n}{\theta^n (n+2)} \left[\theta^{n+2} - 0\right]     \\
     & = \frac{n \theta^2}{n+2}                                     \\
\end{align*}

Then \eqref{eqn: ex2 SE} becomes:
\begin{align*}
    SE
     & = \sqrt{\E[\thetahat_n^2] - \E[\thetahat_n]^2}                                                                   \\
     & = \sqrt{\frac{n \theta^2}{n+2} - \left[ \frac{n \theta}{n+1} \right]^2}                                          \\
     & = \sqrt{n \theta^2 \left[ \frac{1}{n+2} - \frac{n}{(n+1)^2} \right]}                                             \\
     & = \sqrt{n \theta^2 \left[ \frac{1}{n+2} - \frac{n}{n^2 + 2n + 1} \right]}                                        \\
     & = \sqrt{n \theta^2 \left[ \frac{n^2 + 2n + 1}{(n+2)(n^2 + 2n + 1)} - \frac{n(n+2)}{(n^2 + 2n + 1)(n+2)} \right]} \\
     & = \sqrt{n \theta^2 \frac{n^2 + 2n + 1 - n(n+2)}{(n^2 + 2n + 1)(n+2)} }                                           \\
     & = \sqrt{\frac{n \theta^2}{(n^2 + 2n + 1)(n+2)} }                                                                 \\
     & = \frac{\theta}{n+1} \sqrt{\frac{n}{n+2}}                                                                        \\
\end{align*}


\paragraph{What is \(MSE\), the Mean-Square Error of \(\hat\theta_n\) ?\\}
The MSE is given by:
\begin{align*}
    MSE
     & := bias^2 (\thetahat_n) + Var(\thetahat_n)                                          \\
     & = \left[ \E[\thetahat] - \theta \right]^2 + \frac{n \theta^2}{(n+1)^2 (n+2)}        \\
     & = \left[ \frac{n \theta}{n+1} - \theta \right]^2 + \frac{n \theta^2}{(n+1)^2 (n+2)} \\
     & = \left[ \frac{-\theta}{n+1} \right]^2 + \frac{n \theta^2}{(n+1)^2 (n+2)}           \\
     & = \frac{\theta^2}{(n+1)^2} \left[ 1 + \frac{n}{n+2} \right]                         \\
     & = \frac{\theta^2}{(n+1)^2} \frac{2n+2}{n+2}                                         \\
     & = \frac{2 \theta^2}{(n+1)(n+2)}                                                     \\
\end{align*}


\subsection{Chapter 6 - Exercise 3}

Let \(X_1, \ldots, X_n\) be i.i.d. random variables with distribution \(\mathcal{U}(0, \theta)\). \\
Let \(\hat\theta_n := 2 \bar{X_n}\).

\paragraph{What is the bias of \(\hat\theta_n\) ?\\}
\(\hat\theta_n\) is unbiased if \(\E[\hat\theta_n] = \theta\). \\
We compute \(\E[\thetahat_n]\):
\begin{align*}
    \E[\thetahat_n]
     & = \E \left[2\bar{X}_n \right]                                       \\
     & = \E \left[2\frac{X_1 + \ldots + X_n}{n} \right]                    \\
     & = \frac{2}{n} \E \left[X_1 + \ldots + X_n \right]                   \\
     & = \frac{2}{n} \E \left[X_1 \right] + \ldots + \E \left[ X_n \right] \\
     & = 2\E \left[X_1 \right]                                             \\
     & = \theta                                                            \\
\end{align*}
Therefore \(\thetahat\) is unbiased.


\paragraph{What is \(SE\), the standard error of \(\hat\theta_n\) ?\\}

\begin{align*}
    SE
     & = SE(\thetahat_n)                                                                                                         \\
     & = \sqrt{Var(\thetahat_n)}                                                                                                 \\
     & = \sqrt{Var(2\bar{X}_n)}                                                                                                  \\
     & = \frac{2}{n} \sqrt{Var \left( X_1 + \ldots + X_n \right)}                                                                \\
     & = \frac{2}{n} \sqrt{Var \left( X_1 \right) + \ldots +  Var \left( X_n \right)} & \textit{(The } X_i \textit{ are i.i.d.)} \\
     & = \frac{2}{n} \sqrt{\frac{n\theta^2}{12}}                                                                                 \\
     & = \frac{2 \theta}{2\sqrt{3n}}                                                                                             \\
     & = \frac{\theta}{\sqrt{3n}}                                                                                                \\
\end{align*}



\paragraph{What is \(MSE\), the Mean-Square Error of \(\hat\theta_n\) ?\\}
The MSE is given by:
\begin{align*}
    MSE
     & := bias^2 (\thetahat_n) + Var(\thetahat_n)                                        \\
     & = \underbrace{\left[ \E[\thetahat] - \theta \right]^2}_{=0} + \frac{\theta^2}{3n} \\
     & = \frac{\theta^2}{3n}                                                             \\
\end{align*}


\section{Homework for October 20, 2021}
\subsection{Chapter 7 - Exercise 2}

Let $X_{1}, \ldots, X_{n} \sim \operatorname{Bernoulli}(p)$ and let $Y_{1}, \ldots, Y_{m} \sim \operatorname{Bernoulli}(q)$.
\begin{itemize}
    \item Find the plug-in estimator and estimated standard error for $p$.
    \item Find an approximate 90 percent confidence interval for $p$.
    \item Find the plug-in estimator and estimated standard error for $p-q$.
    \item Find an approximate 90 percent confidence interval for $p-q$.
\end{itemize}


\paragraph{Find the plug-in estimator and estimated standard error for $p$.\\}
Let \(\phi\) be the plug-in estimator for \(p\), it is given by:
\[
    \phi = \E[X_i], \ i = 1, \ldots, n
\]
and
\[
    \hat \phi = \E[Z], \text{ with } \P (Z = X_i \mid X_1, \ldots, X_n) = \frac{1}{n}
\]
\begin{align*}
    \hat \phi
     & = \E[Z]                        \\
     & = \sum_{i=1}^n X_i \P(Z = X_i) \\
     & = \sum_{i=1}^n \frac{1}{n} X_i \\
     & = \overline X                  \\
\end{align*}
and the standard error \(se\) is given by:
\begin{align*}
    se (\phi)
     & = \sqrt{Var(\phi)}                   \\                                                                                                                              \\
     & = \sqrt{Var\left(\overline X\right)} \\                                                                                                                              \\
     & = \sqrt{\frac{p(1-p)}{n}}            \\                                                                                                                              \\
\end{align*}


\paragraph{Find an approximate 90 percent confidence interval for $p$.\\}
We know that 90\% (\ie \(\alpha = 0.05\)) confidence intervals are of the following form:
\begin{align*}
     & \overline{X} \pm z_{\alpha/2} se(p_{pin})                                                      \\
     & = \overline{X} \pm 1.645 \sqrt{\frac{1}{n} \left[ \sum_{i=1}^n X_i^2 \right] - \overline{X}^2} \\
\end{align*}


\paragraph{Find the plug-in estimator and estimated standard error for $p-q$.\\}
Let \(\Pi\) be the plug-in estimator for \(p-q\) and \(\chi \in \left\{ X_1, \ldots, X_n, Y_1, \ldots, Y_m\right\}\)
\begin{align*}
    \P(\Pi = \chi)
     & = \frac{1}{\#\left\{ X_1, \ldots, X_n\right\} + \#\left\{ Y_1, \ldots, Y_m\right\}} \\
     & = \frac{1}{m+n}
\end{align*}
\begin{align*}
    se (\Pi)
     & = \sqrt{Var(\Pi)}                                                                                                                           \\
     & = \sqrt{\E[\Pi^2] - \E[\Pi]^2}                                                                                                              \\
     & = \sqrt{\left[\sum_{i=1}^n \chi^2 \P \left\{ \Pi = \chi \right\}\right] - \left[\sum_{i=1}^n \chi \P \left\{ \Pi = \chi \right\} \right]^2} \\
     & = \sqrt{\left[\sum_{i=1}^n \chi^2 \frac{1}{m+n}\right] - \left[\sum_{i=1}^n \chi \frac{1}{m+n} \right]^2}                                   \\
     & = \sqrt{\frac{1}{m+n}\left[\sum_{i=1}^n \chi^2\right] - \overline{\chi}^2}                                                                  \\
\end{align*}

\paragraph{Find an approximate 90 percent confidence interval for $p-q$.\\}
We know that 90\% (\ie \(\alpha = 0.05\)) confidence intervals are of the following form:
\begin{align*}
     & \overline{\chi} \pm z_{\alpha/2} se(p_{pin})                                                         \\
     & = \overline{\chi} \pm 1.645 \sqrt{\frac{1}{m+n}\left[\sum_{i=1}^n \chi^2\right] - \overline{\chi}^2} \\
\end{align*}

\subsection{Chapter 7 - Exercise 5}
\paragraph{Let $x$ and $y$ be two distinct points. Find $\operatorname{Cov}\left(\hat{F}_{n}(x), \hat{F}_{n}(y)\right)$.}
\subsection{Chapter 7 - Exercise 6}

\section{Homework for October 29}
\subsection{Custom exercise}
Let \(N=50\), \(Y_1, \ldots, Y_n\) are i.i.d. \(\mathcal{N}(0,1)\).
Let \(X_i = e^{Y_i}\).\\
Let \(\theta = \operatorname{skewness}(X) = (e+2)\sqrt{e-1}\) (\(X\) is log normal distributed) \\
\paragraph{Compute the 3 types of normal confidence intervals for \(\theta\).}
\paragraph{Repeat the experiment to check how often \(\theta\) belongs to the confidence intervals.}


\section{Homework for November 5, 2021}
\subsection{Chapter 9 - Exercise 1}
\paragraph{Let $X_{1}, \ldots, X_{n} \sim \operatorname{Gamma}(\alpha, \beta)$. Find the method of moments estimator for $\alpha$ and $\beta$.}
From tables, we have that:
\begin{equation*}
    \begin{cases}
        \E [X_i] = \frac{\alpha}{\beta} \\
        Var (X_i) = \frac{\alpha}{\beta^2}
    \end{cases}
\end{equation*}
Equating with empirical expected value and empirical variance respectively.
\begin{align*}
             &
    \begin{cases}
        \frac{1}{n}\sum_{i=1}^n X_i = \E [X_i] \\
        \frac{1}{n}\sum_{i=1}^n (\bar X - X_i)^2 = Var (X_i)
    \end{cases} \\
    \implies &
    \begin{cases}
        \frac{1}{n}\sum_{i=1}^n X_i = \frac{\hat\alpha}{\hat\beta} \\
        \frac{1}{n}\sum_{i=1}^n (\bar X - X_i)^2 = \frac{\hat\alpha}{\hat\beta^2}
    \end{cases} \\
    \implies &
    \begin{cases}
        \hat\alpha = \frac{\hat\beta}{n}\sum_{i=1}^n X_i \\
        \frac{1}{n}\sum_{i=1}^n (\bar X - X_i)^2 = \frac{\hat\alpha}{\hat\beta^2}
    \end{cases} \\
    \implies &
    \begin{cases}
        \hat\alpha = \frac{\hat\beta}{n}\sum_{i=1}^n X_i \\
        \frac{1}{n}\sum_{i=1}^n (\bar X - X_i)^2 = \frac{1}{\hat\beta^2} \frac{\hat\beta}{n}\sum_{i=1}^n X_i
    \end{cases} \\
    \implies &
    \begin{cases}
        \hat\alpha = \frac{\hat\beta}{n}\sum_{i=1}^n X_i \\
        \frac{1}{n}\sum_{i=1}^n (\bar X - X_i)^2 = \frac{1}{\hat\beta n}\sum_{i=1}^n X_i
    \end{cases} \\
    \implies &
    \begin{cases}
        \hat\alpha = \frac{\beta}{n}\sum_{i=1}^n X_i \\
        \hat\beta = \frac{\sum_{i=1}^n X_i}{\sum_{i=1}^n (\bar X - X_i)^2}
    \end{cases} \\
    \implies &
    \begin{cases}
        \hat\alpha = \frac{1}{n}\frac{\left[\sum_{i=1}^n X_i\right]^2}{\sum_{i=1}^n (\bar X - X_i)^2} \\
        \hat\beta = \frac{\sum_{i=1}^n X_i}{\sum_{i=1}^n (\bar X - X_i)^2}
    \end{cases} \\
\end{align*}


\subsection{Chapter 9 - Exercise 2}
\paragraph{Let $X_{1}, \ldots, X_{n} \sim$ Uniform $(a, b)$ where $a$ and $b$ are unknown parameters and $a<b$.}
\begin{itemize}
    \item (a) Find the method of moments estimators for $a$ and $b$.
    \item (b) Find the MLE $\hat{a}$ and $\hat{b}$.
    \item (c) Let $\tau=\int x d F(x)$. Find the MLE of $\tau$.
    \item (d) Let $\hat{\tau}$ be the MLE of $\tau$. Let $\widetilde{\tau}$ be the nonparametric plug-in estimator of $\tau=\int x d F(x)$. Suppose that $a=1, b=3$, and $n=10$. Find the MSE of $\hat{\tau}$ by simulation. Find the MSE of $\widetilde{\tau}$ analytically. Compare.
\end{itemize}

\paragraph{(a) Find the method of moments estimators for $a$ and $b$.\\}
We compute the first order of moments:
\begin{align}
    \nonumber
    \E[X_1]
     & = \int_a^b x f(x) dx                            \\
    \nonumber
     & = \frac{1}{b-a} \int_a^b x dx                   \\
    \nonumber
     & = \frac{1}{b-a} \left[ \frac{x^2}{2}\right]_a^b \\
    \nonumber
     & = \frac{1}{b-a} \frac{b^2 - a^2}{2}             \\
    \label{ch9 ex2: first moment}
     & = \frac{a + b}{2}
\end{align}
Now compute the second order of moments:
\begin{align}
    \nonumber
    \E[X_1^2]
     & = \int_{a}^{b} x^2 f (x) dx                                                          \\
    \nonumber
     & = \int_{a}^{b} x^2 \frac{1}{b-a} dx                                                  \\
    \nonumber
     & = \frac{1}{b-a} \left[ \frac{x^3}{3} \right]_{a}^{b}                                 \\
    \nonumber
     & = \frac{1}{b-a} \left[ \frac{b^3 - a^3}{3} \right]                                   \\
    \nonumber
     & = \frac{1}{b-a} \left[\frac{\left(b - a\right)\left(a^2 + ab + b^2\right)}{3}\right] \\
    \label{ch9 ex2: second moment}
     & = \frac{a^2 + ab + b^2}{3}
\end{align}
Now on the one hand we equate \eqref{ch9 ex2: first moment} with \(\hat{\mu_1} := \frac{1}{n} \sum_{i=1}^n X_i\). On the other hand, we equate \eqref{ch9 ex2: second moment} with \(\hat{\mu_2} := \frac{1}{n}\sum_{i=1}^n X_i^2\). Therefore we get a system of equations:
\begin{align}
    \nonumber
     & \begin{cases}
        \hat\mu_1 = \frac{a + b}{2} \\
        \hat\mu_2 = \frac{a^2 + ab + b^2}{3}
    \end{cases} \\
    \nonumber
     & \begin{cases}
        b = 2\hat\mu_1 - a \\
        3\hat\mu_2 = a^2 + a\left[2\hat\mu_1 - a\right] + \left[2\hat\mu_1 - a\right]^2
    \end{cases} \\
    \nonumber
     & \begin{cases}
        b = 2\hat\mu_1 - a \\
        3\hat\mu_2 = a^2 + 2\hat\mu_1 a - a^2 + 4\hat\mu_1^2 - 4\hat\mu_1 a + a^2
    \end{cases} \\
    \label{ch9 ex2: system solve a}
     & \begin{cases}
        b = 2\hat\mu_1 - a \\
        a^2 - 2\hat\mu_1 a + (4\hat\mu_1^2 - 3\hat\mu_2) = 0
    \end{cases}
\end{align}
Then, the second equation of \eqref{ch9 ex2: system solve a} yields:
\begin{align*}
    a_1
     & = \frac{2\hat\mu_1 - \sqrt{(2\hat\mu_1)^2 - 4\left(4\hat\mu_1^2 - 3\hat\mu_2\right)}}{2} \\
     & = \hat\mu_1 - \sqrt{\hat\mu_1^2 - 4\hat\mu_1^2 + 3\hat\mu_2}                             \\
     & = \hat\mu_1 - \sqrt{3\left(\hat\mu_2 - \hat\mu_1^2\right)}                               \\
\end{align*}
and
\begin{align*}
    a_1
     & = \frac{2\hat\mu_1 + \sqrt{(2\hat\mu_1)^2 - 4\left(4\hat\mu_1^2 - 3\hat\mu_2\right)}}{2} \\
     & = \hat\mu_1 + \sqrt{\hat\mu_1^2 - 4\hat\mu_1^2 + 3\hat\mu_2}                             \\
     & = \hat\mu_1 + \sqrt{3\left(\hat\mu_2 - \hat\mu_1^2\right)}                               \\
\end{align*}
Let \(b_1, b_2\) be associated with \(a_1, a_2\) respectively. Then the first equation of \eqref{ch9 ex2: system solve a} becomes:
\begin{align*}
     & \begin{cases}
        b_1 := 2\hat\mu_1 - \left(\hat\mu_1 - \sqrt{3\left(\hat\mu_2 - \hat\mu_1^2\right)}\right) \\
        b_2 := 2\hat\mu_1 - \left(\hat\mu_1 + \sqrt{3\left(\hat\mu_2 - \hat\mu_1^2\right)}\right)
    \end{cases} \\
     & \begin{cases}
        b_1 := \hat\mu_1 + \sqrt{3\left(\hat\mu_2 - \hat\mu_1^2\right)} \\
        b_2 := \hat\mu_1 - \sqrt{3\left(\hat\mu_2 - \hat\mu_1^2\right)} \\
    \end{cases} \\
\end{align*}
Since \(b_2 > a_2\), which is impossible by the exercise, we have that the method of moment estimators are:
\[
    \begin{cases}
        a = \hat\mu_1 - \sqrt{3\left(\hat\mu_2 - \hat\mu_1^2\right)} \\
        b = \hat\mu_1 + \sqrt{3\left(\hat\mu_2 - \hat\mu_1^2\right)}
    \end{cases}
\]


\paragraph{(b) Find the MLE $\hat{a}$ and $\hat{b}$.\\}
Let \(\theta := \left(a,b\right) \in \Theta \subseteq \R^2\). We define \(\L\) the likelihood function as follows:
\begin{align*}
    \L (\theta)
     & = f(X_1, \ldots, X_n \mid \theta)  \\
     & = \prod_{i=1}^n f(X_i \mid \theta) \\
\end{align*}
Now we want to find \(\hat\theta := (\hat a, \hat b)\) the argument maximizing the likelihood function
\begin{align*}
    \hat\theta
     & := (\hat a, \hat b)                \\
     & := \arg \max_\Theta \L(\theta)     \\
     & = \arg \max_\Theta \log \L(\theta) \\
\end{align*}
therefore we have
\begin{align*}
    \log \L(\theta)
     & = \log \prod_{i=1}^n f(X_i \mid \theta) \\
     & = \sum_{i=1}^n \log f(X_i \mid (a,b))   \\
     & = \sum_{i=1}^n \log \frac{1}{b-a}       \\
     & = -n \log (b-a)                         \\
\end{align*}
hence
\begin{equation*}
    \begin{cases}
        \frac{\partial}{\partial a} \log \L (\theta) = \frac{n}{b-a}  \\
        \frac{\partial}{\partial b} \log \L (\theta) = -\frac{n}{b-a} \\
    \end{cases}
\end{equation*}
Now we have that
\begin{align*}
             & \begin{cases}
        \frac{\partial}{\partial a} \log \L (\theta) > 0 \\
        \frac{\partial}{\partial b} \log \L (\theta) < 0 \\
    \end{cases} \\
    \implies &
    \begin{cases}
        \log \L (\theta) \text{ is increasing with respect to } a \\
        \log \L (\theta) \text{ is decreasing with respect to } b \\
    \end{cases}            \\
    \implies &
    \begin{cases}
        \hat a = \min \left\{X_1, \ldots, X_n\right\} \\
        \hat b = \max \left\{X_1, \ldots, X_n\right\} \\
    \end{cases}            \\
\end{align*}


\paragraph{(c) Let $\tau=\int x d F(x)$. Find the MLE of $\tau$.\\}
Let \(\theta := \tau \in \Theta \subseteq \R\). We have the following:
\begin{align*}
    \tau
     & = \int x d F(x)               \\
     & = \int x \frac{d F(x)}{dx} dx \\
     & = \int x f(x) dx              \\
     & = \E[X_1]                     \\
     & = \frac{a + b}{2}             \\
\end{align*}
Hence
\begin{equation*}
    \hat\tau = \frac{\hat a + \hat b}{2}
\end{equation*}



\paragraph{(d) Let $\hat{\tau}$ be the MLE of $\tau$. Let $\widetilde{\tau}$ be the nonparametric plug-in estimator of $\tau=\int x d F(x)$. Suppose that $a=1, b=3$, and $n=10$. Find the MSE of $\hat{\tau}$ by simulation. Find the MSE of $\widetilde{\tau}$ analytically. Compare.\\}
The MSE is defined by:
\begin{align*}
    MSE (\tilde \tau)
     & := Var(\tilde\tau)   + bias^2 (\tilde \tau)                                          \\
     & = \E[\tilde\tau^2] - \E[\tilde\tau]^2 + \left[\E[\tilde\tau] - \tilde\tau\right]^2   \\
     & = \E[\bar X^2] - \E[\bar X]^2 + \underbrace{\left[\E[\bar X] - \bar X\right]^2}_{=0} \\
     & = \frac{1}{n}\left[\frac{a^2 + ab + b^2}{3} - \left[\frac{a+b}{2}\right]^2\right]    \\
     & = \frac{1}{n}\left[\frac{a^2 + ab + b^2}{3} - \frac{a^2 + 2ab + b^2}{4}\right]       \\
     & = \frac{1}{n}\left[\frac{a^2 + ab + b^2 - 3a^2 - 6ab - 3b^2}{12}\right]              \\
     & = \frac{1}{n}\left[\frac{4a^2 + 4ab + 4b^2 - 3a^2 - 6ab - 3b^2}{12}\right]           \\
     & = \frac{1}{n}\left[\frac{a^2 -2ab + b^2}{12}\right]                                  \\
     & = \frac{1}{n}\left[\frac{(a-b)^2}{12}\right]                                         \\
     & = \frac{(a-b)^2}{12n}                                                                \\
\end{align*}

\section{Preparation for mid-term}
\subsection{Chapter 9 - Exercises 5}

\paragraph{Let $X_{1}, \ldots, X_{n} \sim$ Poisson $(\lambda)$. Find the method of moments estimator, the maximum likelihood estimator and the Fisher information $I(\lambda)$.\\}

We know that the first moment for a Poisson distribution $\mu_1 = \lambda$. We want to evaluate \(\hat\theta := \lambda\).
We set \(\mu_1 = \bar{X}\), which gives:
\begin{align*}
     & \mu_1 = \bar{X}
    \implies
    \hat \theta = \bar{X}
\end{align*}
Now we want to find the maximum likelihood estimator.
\begin{align*}
    \L(\theta)
     & = f(X_1, \ldots, X_n \mid \theta)                 \\
     & = \prod_{i=1}^n f(X_i\mid \theta)                 \\
     & = \prod_{i=1}^n \frac{\lambda^k e^{-\lambda}}{k!} \\
\end{align*}
[NOT FINISHED]

\subsection{Chapter 9 - Exercises 6}
\paragraph{Let $X_{1}, \ldots, X_{n} \sim N(\theta, 1)$. Define
    $$
        Y_{i}=\left\{\begin{array}{ll}
            1 & \text { if } X_{i}>0      \\
            0 & \text { if } X_{i} \leq 0
        \end{array}\right.
    $$
    Let $\psi=\mathbb{P}\left(Y_{1}=1\right)$.}
\paragraph{(a) Find the maximum likelihood estimator $\widehat{\psi}$ of $\psi$.}
\paragraph{(b) Find an approximate 95 percent confidence interval for $\psi$.}
\paragraph{(c) Define $\widetilde{\psi}=(1 / n) \sum_{i} Y_{i}$. Show that $\widetilde{\psi}$ is a consistent estimator of $\psi$.}
\paragraph{(d) Compute the asymptotic relative efficiency of $\widetilde{\psi}$ to $\widehat{\psi}$. Hint: Use the delta method to get the standard error of the MLE. Then compute the standard error (i.e. the standard deviation) of $\widetilde{\psi}$.}
\paragraph{(e) Suppose that the data are not really normal. Show that $\widehat{\psi}$ is not consistent. What, if anything, does $\widehat{\psi}$ converge to?}

\subsection{Chapter 9 - Examples 20}
\subsection{Chapter 9 - Examples 21}
\paragraph{Let $X_{1}, \ldots, X_{n} \sim N\left(\theta, \sigma^{2}\right)$ where $\sigma^{2}$ is known. The score function is $s(X ; \theta)=(X-\theta) / \sigma^{2}$ and $s^{\prime}(X ; \theta)=-1 / \sigma^{2}$ so that $I_{1}(\theta)=1 / \sigma^{2} .$ The MLE is $\widehat{\theta}_{n}=\bar{X}_{n}$. According to Theorem 9.18, $\bar{X}_{n} \approx N\left(\theta, \sigma^{2} / n\right)$. In this case, the Normal approximation is actually exact.}

\subsection{Chapter 9 - Examples 22}
\paragraph{Let $X_{1}, \ldots, X_{n} \sim \operatorname{Poisson}(\lambda) .$ Then $\widehat{\lambda}_{n}=\bar{X}_{n}$ and some calculations show that $I_{1}(\lambda)=1 / \lambda$, so
    $$
        \widehat{\mathrm{se}}=\frac{1}{\sqrt{n I\left(\widehat{\lambda}_{n}\right)}}=\sqrt{\frac{\widehat{\lambda}_{n}}{n}}
    $$
    Therefore, an approximate $1-\alpha$ confidence interval for $\lambda$ is $\widehat{\lambda}_{n} \pm z_{\alpha / 2} \sqrt{\widehat{\lambda}_{n} / n}$.}

\section{In class exercise December 3, 2021}
\subsection{Exercise 1}
\paragraph{Let \(X_1, \ldots, X_N \sim \mathcal{N}(\mu, \sigma^2)\) with \(\mu\) given.}
\paragraph{i) Compute \(\hat{\sigma}_{ML}\) and estimator \(se(\hat{\sigma}_{ML})\)}
\begin{align*}
    \log \L(\sigma^2)
     & = \sum_{i=1}^N \log f_\sigma (X_i)                                                                                     \\
     & = \sum_{i=1}^N \left[\log \frac{1}{\sigma \sqrt{2 \pi}}\right] e^{-\frac{1}{2} \left(\frac{X - \mu}{\sigma}\right)^2}  \\
     & = \left[N \log \frac{1}{\sigma \sqrt{2 \pi}}\right] - \frac{1}{2} \sum_{i=1}^N \left(\frac{X_i - \mu}{\sigma}\right)^2 \\
     & = - N (\log \sigma \sqrt{2 \pi}) - \frac{1}{2} \sum_{i=1}^N \left(\frac{X_i - \mu}{\sigma}\right)^2                    \\
     & = - N (\log \sigma  + \log \sqrt{2 \pi}) - \frac{1}{2} \sum_{i=1}^N \left(\frac{X_i - \mu}{\sigma}\right)^2            \\
\end{align*}

\begin{align*}
    \frac{\partial}{\partial \sigma} \log \L(\sigma^2)
     & = \frac{\partial}{\partial \sigma} \left[ - N (\log \sigma  + \log \sqrt{2 \pi}) - \frac{1}{2} \sum_{i=1}^N \left(\frac{X_i - \mu}{\sigma}\right)^2 \right] \\
     & = - \frac{N}{\sigma} - \frac{1}{2} \sum_{i=1}^n \left(X_i - \mu\right)^2 \frac{\partial}{\partial \sigma} \sigma^{-2}                                       \\
     & = - \frac{N}{\sigma} - \frac{1}{2} \sum_{i=1}^n \left(X_i - \mu\right)^2 (-2)\sigma^{-3}                                                                    \\
     & = - \frac{N}{\sigma} +\sum_{i=1}^n \left(X_i - \mu\right)^2 \sigma^{-3}                                                                                     \\
\end{align*}

\begin{align*}
             & \frac{\partial}{\partial \hat \sigma_{ML}} \log \L(\hat \sigma_{ML}^2) = 0                    \\
    \implies & - \frac{N}{\hat \sigma_{ML}} +\sum_{i=1}^n \left(X_i - \mu\right)^2 \hat \sigma_{ML}^{-3} = 0 \\
    \implies & \hat \sigma_{ML}^{-2} = \frac{N}{\sum_{i=1}^n \left(X_i - \mu\right)^2}                       \\
    \implies & \hat \sigma_{ML}^{2} = \frac{\sum_{i=1}^n \left(X_i - \mu\right)^2}{N}                        \\
    \implies & \hat \sigma_{ML} = \sqrt{\frac{\sum_{i=1}^n \left(X_i - \mu\right)^2}{N}}                     \\
\end{align*}

\begin{align*}
    se (\hat \sigma_{ML}) & = \frac{1}{I_N (\sigma)}
\end{align*}
\begin{align*}
    \frac{\partial^2}{\partial \sigma^2} \left[ \log \L (\sigma) \right]
     & = \frac{\partial}{\partial \sigma} \left[ - \frac{N}{\sigma} +\sum_{i=1}^n \left(X_i - \mu\right)^2 \sigma^{-3} \right] \\
     & =  \frac{N}{\sigma^2} -3 \sum_{i=1}^n \left(X_i - \mu\right)^2 \sigma^{-4}                                              \\
\end{align*}
\begin{align*}
    I_N(\sigma)
     & = - \E \left[ \frac{\partial^2}{\partial \sigma^2} \log \L (\sigma) \right]                            \\
     & = - \E \left[ \frac{N}{\sigma^2} -3 \sum_{i=1}^n \left(X_i - \mu\right)^2 \sigma^{-4} \right]          \\
     & = - \frac{N}{\sigma^2} - \frac{-3}{\sigma^{4}} \sum_{i=1}^n \E \left[ \left(X_i - \mu\right)^2 \right] \\
     & = - \frac{N}{\sigma^2} - \frac{-3}{\sigma^{4}} \sum_{i=1}^n Var \left[ X_i ^2 \right]                  \\
     & = - \frac{N}{\sigma^2} - \frac{-3}{\sigma^{4}} \sum_{i=1}^n \sigma^2                                   \\
     & = - \frac{N}{\sigma^2} - \frac{-3N}{\sigma^{2}}                                                        \\
     & = \frac{2N}{\sigma^{2}}                                                                                \\
\end{align*}

Thus
\[
    se(\hat \sigma_{ML}) = \frac{1}{\sqrt{I_N (\sigma)}} = \frac{\sigma}{\sqrt{2N}}
\]

\paragraph{ii) Compute \(\hat{\sigma}_{ML}\) and estimator \(se(\hat{\sigma}_{ML})\)}

\section{Homework for December 10, 2021}
\subsection{Chapter 9 - Exercise 3}

3. Let $X_{1}, \ldots, X_{n} \sim N\left(\mu, \sigma^{2}\right)$. \\
Let $\tau$ be the $.95$ percentile, i.e. $\mathbb{P}(X<\tau)=$ $0.95$ \\
(a) Find the MLE of $\tau$. \\
(b) Find an expression for an approximate $1-\alpha$ confidence interval for $\tau$. \\
(c) Suppose the data are: \\
$$
    \begin{array}{rrrrrr}
        3.23 & -2.50 & 1.88  & -0.68 & 4.43  & 0.17 \\
        1.03 & -0.07 & -0.01 & 0.76  & 1.76  & 3.18 \\
        0.33 & -0.31 & 0.30  & -0.61 & 1.52  & 5.43 \\
        1.54 & 2.28  & 0.42  & 2.33  & -1.03 & 4.00 \\
        0.39 &       &       &       &       &
    \end{array}
$$
Find the MLE $\widehat{\tau}$. Find the standard error using the delta method. Find the standard error using the parametric bootstrap.

\subsubsection{(a)}
\begin{align*}
             & \P \left( X_i < \tau \right) = 0.95                                                                         \\
    \implies &
    \P \left( \underbrace{\frac{X_i - \mu}{\sigma}}_{:=Z \sim \mathcal{N}(0,1)} < \frac{\tau - \mu}{\sigma} \right) = 0.95 \\
    \implies &
    \underbrace{\P \left( Z < \frac{\tau - \mu}{\sigma} \right)}_{:= \Phi (\frac{\tau - \mu}{\sigma})} = 0.95              \\
    \implies &
    \Phi \left(\frac{\tau - \mu}{\sigma}\right) = 0.95                                                                     \\
    \implies &
    \frac{\tau - \mu}{\sigma} = \Phi^{-1}\left(0.95\right)                                                                 \\
    \implies &
    \tau = \sigma \Phi^{-1}\left(0.95\right) + \mu                                                                         \\
    \implies &
    \hat \tau_{ML} = \hat\sigma_{ML} \Phi^{-1}\left(0.95\right) + \hat\mu_{ML}                                             \\
\end{align*}

\subsubsection{(b)}
Let \(\tau = \sigma \Phi^{-1}\left(0.95\right) + \mu := g(\mu, \sigma)\), so \(\hat\tau_{ML} = g(\hat\mu_{ML}, \hat\sigma_{ML})\), by equivariance of ML.

Therefore:
\begin{align*}
    \hat\tau_{ML} - \tau
     & = g(\hat\mu_{ML}, \hat\sigma_{ML}) - g(\mu, \sigma)                                          \\
     & \approx \nabla g(\mu, \sigma) \left[ (\hat\mu_{ML}, \hat\sigma_{ML}) - (\mu, \sigma) \right]
\end{align*}

\subsubsection{(c)}



\section{Simple Linear Regression Exercises}

\subsection{Exercise 1}
\subsubsection{(a)}
Recall how the OLS estimator
$$
    \hat{\beta}=\left(X^{T} X\right)^{-1} X^{T} Y
$$
is computed and show that the fitted values $\hat{Y}$ are obtained as
$$
    \hat{Y}=X\left(X^{T} X\right)^{-1} X^{T} Y
$$

\subsubsection{(b)}
(b) Show that that $\hat{\beta}$ is an unbiased estimator of $\beta$ and compute its variance.
\begin{align*}
    \E [\hat \beta]
     & = \E \left[ \left(X^{T} X\right)^{-1} X^{T} Y \right]                     \\
     & = \left(X^{T} X\right)^{-1} X^{T} \E \left[ Y \right]                     \\
     & = \left(X^{T} X\right)^{-1} X^{T} \E \left[ X \beta + \varepsilon \right] \\
     & = \left(X^{T} X\right)^{-1} X^{T} X \E \left[ \beta \right]               \\
     & = \left(X^{T} X\right)^{-1} X^{T} X \E \left[ \beta \right]               \\
     & =  \beta                                                                  \\
\end{align*}
So \(\hat \beta\) is unbiased.

\begin{align*}
    Var(\hat \beta)
     & = \E \left[ \left( \hat \beta - \E[\hat \beta ]\right) \left( \hat \beta - \E[\hat \beta]\right)^T \right]                            \\
     & = \E \left[ \left( \hat \beta - \beta \right) \left( \hat \beta - \beta \right)^T \right]                                             \\
     & = \E \left[ \left( \left(X^T X \right)^{-1} X^T \varepsilon \right) \left( \left(X^T X \right)^{-1} X^T \varepsilon \right)^T \right] \\
     & = \E \left[ \left(X^T X \right)^{-1} X^T \varepsilon \varepsilon^T X \left(\left(X^T X \right)^{-1} \right)^T \right]                 \\
     & = \left(X^T X \right)^{-1} X^T \cdot \E \left[ \varepsilon \varepsilon^T \right] \cdot X \left(\left(X^T X \right)^{-1} \right)^T     \\
     & = \left(X^T X \right)^{-1} X^T \cdot \sigma^2 I_n \cdot X \left(X^T X \right)^{-1}                                                    \\
     & = \sigma^2 \left(X^T X \right)^{-1}                                                                                                   \\
\end{align*}

\subsubsection{(c)}
We now introduce the matrix $P_{1}:=X\left(X^{T} X\right)^{-1} X^{T}$ and show that it is a projection matrix ${ }^{1}$ In order to do that

\paragraph{1. Show that $P_{1}$ is symmetric.}
\begin{align*}
    P_1^T
     & = \left[ X\left(X^{T} X\right)^{-1} X^{T} \right]^T \\
     & = X\left(X^{T} X\right)^{-1^T} X^{T}                \\
     & = X\left(X^{T} X\right)^{-1} X^{T}                  \\
     & = P_1                                               \\
\end{align*}

\paragraph{2. Show that $P_{1}^{2}=P_{1}$.}
\begin{align*}
    P_1^2
     & = \left[ X\left(X^{T} X\right)^{-1} X^{T} \right]^2                  \\
     & =  X\left(X^{T} X\right)^{-1} X^{T} X\left(X^{T} X\right)^{-1} X^{T} \\
     & =  X\left(X^{T} X\right)^{-1} X^{T}                                  \\
     & =  P_1                                                               \\
\end{align*}

\subsubsection{(d)}
\paragraph{Prove that $X \hat{\beta}=X \beta+P_{1} \varepsilon$}

\begin{align*}
    X \hat \beta
     & = X \left(X^T X \right)^{-1} X^T Y                                    \\
     & = X \left(X^T X \right)^{-1} X^T \left( X \beta + \varepsilon \right) \\
     & = X \beta + X \left(X^T X \right)^{-1} X^T \varepsilon                \\
     & = X \beta + P_1 \varepsilon                                           \\
\end{align*}


\subsubsection{(e)}
\paragraph{Show that the OLS residuals $\hat{\varepsilon}:=Y-X \hat{\beta}$ are obtained as
    $$
        \hat{\varepsilon}=\left(I_{n}-P_{1}\right) Y
    $$}

\begin{align*}
    \hat{\varepsilon}
     & := Y-X \hat{\beta} \\
\end{align*}

\subsubsection{(f)}
\paragraph{Prove that the matrix $P_{2}:=\left(I_{n}-P_{1}\right)$ is also a projection matrix and show that $P_{1} P_{2}=0_{n}$.}

    \subsubsection{(g)}
    \paragraph{Show that $\hat{\varepsilon}=P_{2} \varepsilon$.}

    \begin{align*}
        \hat \varepsilon
         & = y - \hat y                             \\
         & = y - X \left(X^T X \right)^{-1} X^T y   \\
         & = \left[ I - P_1 \right] y               \\
         & = P_2 y                                  \\
         & = P_2 \left(X \beta + \varepsilon\right) \\
         & = P_2 \varepsilon                        \\
    \end{align*}


    \paragraph{Show that \(\E[\hat \varepsilon] = 0\)}

    \begin{align*}
        \E[\hat \varepsilon]
         & = \E \left[ P_2 \varepsilon \right] \\
         & = P_2 \E \left[ \varepsilon \right] \\
         & = 0                                 \\
    \end{align*}

    \paragraph{Compute \(Var(\hat \varepsilon)\)}

    \begin{align*}
        Var(\hat \varepsilon)
         & = \E \left[\hat \varepsilon \hat \varepsilon^T\right] \\
         & = P_2 \E \left[\varepsilon \varepsilon^T\right] P_2^T \\
         & = P_2 \sigma^2 P_2^T                                  \\
         & =\sigma^2 P_2 P_2^T                                   \\
         & =\sigma^2 P_2                                         \\
    \end{align*}


    \begin{align*}
        Cov(\hat \beta, \hat \varepsilon)
         & = \E\left[ \left(\hat \beta - \E \left[\hat \beta \right]\right) \left(\hat \varepsilon - \E \left[\hat \varepsilon \right]\right)^T\right] \\
         & = \E\left[ \left(\hat \beta - \beta\right) \left( P_2 \varepsilon\right)^T\right]                                                           \\
         & = \E\left[ \left( X X \right)^{-1} X^T \left( P_2 \varepsilon\right)^T\right]                                                               \\
         & \vdots                                                                                                                                      \\
         & = 0
    \end{align*}

    \subsection{Exercise 2}

    Let
    \begin{equation}
        \label{eqn: eq1}
        \sum_{i=1}^{n} \hat{\varepsilon}_{i}^{2}=\|\hat{\varepsilon}\|^{2}=\sum_{k=1}^{n-p} Z_{k}^{2},
    \end{equation}
    where $Z_{k} \sim \mathcal{N}\left(0, \sigma^{2}\right)$ and are all independent.

\subsubsection{(a)}
\paragraph{Modify \eqref{eqn: eq1} is such a way to obtain a sum of independent standard (unit variance) Gaussian random variables.}

\begin{align*}
    \|\hat \varepsilon\|^2
     & = \sum_{k=1}^{n-p} Z_k^2 \\
     & = \sigma^2 \underbrace{\sum_{k=1}^{n-p} \frac{Z_k^2}{\sigma^2}}_{\sim \mathcal{N}(0, 1)} \\
\end{align*}
Hence \( \| \varepsilon \|^2 \sim \sigma \chi^2_{n-p} \)

\end{document}