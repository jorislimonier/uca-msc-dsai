\documentclass[12pt]{article}
\usepackage[utf8]{inputenc}

% Importing settings from setup.sty
\usepackage{setup}
\usepackage{booktabs}
\usepackage{multicol}
\usepackage{multirow}
\usepackage{glossaries}
% \makenoidxglossaries
\newcommand{\prox}{\operatorname{prox}}

% \newacronym{bmi}{BMI}{Body Mass Index}


% \pagenumbering{roman}
\begin{document}

% Inserting title page
\import{./}{title}

\pagenumbering{gobble}
\tableofcontents
% \listoffigures
% \listoftables



\newgeometry{
  left=25mm,
  right=25mm,
  top=25mm,
  bottom=25mm}
\pagenumbering{arabic}

\section{Exercise 1}
\begin{equation}
  \prox_{\tau f}(x) = \arg\min_{u} \frac{1}{2\tau}\|u-x\|^2 + f(u)
\end{equation}
\begin{equation}
  \prox_{\tau |\cdot|}(x) = \arg\min_{u} \frac{1}{2\tau}\|u-x\|^2 + |u|
\end{equation}
\begin{equation}
\end{equation}
Let $h(x) = \frac{1}{2\tau}\|u-x\|^2 + |u|$. Then
\begin{align*}
  \frac{\partial}{\partial u} h(x)
   & = \frac{\partial}{\partial u} \frac{1}{2\tau}\|u-x\|^2 + |u| \\
   & =
  \begin{cases}
    \frac{1}{\tau}(u-x) - 1,  & u < 0 \\
    0,                        & u = 0 \\
    \frac{1}{\tau} (u-x) + 1, & u > 0 \\
  \end{cases}
\end{align*}

\paragraph{Case $u > 0$.}
\begin{align*}
           &
  \frac{\partial}{\partial u} h(x) = 0 \\
  \implies &
  \frac{1}{\tau} (u-x) + 1 = 0 \\
  \implies &
  u = x - \tau                 \\
\end{align*}

\paragraph{Case $u < 0$.}
\begin{align*}
           &
  \frac{\partial}{\partial u} h(x) = 0 \\
  \implies &
  \frac{1}{\tau} (u-x) - 1 = 0 \\
  \implies &
  u = x + \tau                 \\
\end{align*}

\paragraph{Case $u = 0$.}
$\partial h(u) = [-1, 1]$
Therefore
\begin{align*}
  \prox_{\tau |\cdot|}(x)
   & = \begin{cases}
    x - \tau, & x < -\tau              \\
    0,        & -\tau \leq x \leq \tau \\
    x + \tau, & x > \tau               \\
  \end{cases}
\end{align*}

\section{Exercise 2}

\begin{equation}
  f(x) = |x|_0 = \begin{cases}
    0, & x = 0    \\
    1, & x \neq 0
  \end{cases}
\end{equation}
\begin{equation}
  \prox_{\tau |\cdot|_0}(x) = \arg\min_{u} \frac{1}{2\tau}\|u-x\|^2 + |u|_0
\end{equation}

\begin{equation}
  \prox_{\tau |\cdot|_0}(x) = \begin{cases}
    0, & x = 0    \\
    x, & x \neq 0
  \end{cases}
\end{equation}

\begin{align*}
  h'(x) = \begin{cases}
    0, & x \neq 0 \\
    1, & x = 0
  \end{cases}
\end{align*}
So it is better to choose $u = 0$ when $\frac{x^2}{2\tau} < 1$, else, set $u = x$.

\section{Exercise 3}
\begin{equation}
  \delta_{\R_+^n} (x) = \begin{cases}
    \infty, & x \not \in \R_+^n \\
    0,      & \text{otherwise}
  \end{cases}
\end{equation}
\begin{equation}
  \prox_{\tau |\cdot|_1 + \delta_{\R_+^n} (\cdot)}(x) = \arg\min_{u} \frac{1}{2\tau}\|u-x\|^2 + |u|_1 + \delta_{\R_+^n} (u)
\end{equation}

\begin{equation}
  \prox_{\tau |\cdot|_1 + \delta_{\R_+^n} (\cdot)}(x) = \arg\min_{u} \frac{1}{2\tau}\|u-x\|^2 + |u|_1 + \delta_{\R_+^n} (u)
\end{equation}
Given by prof:
\begin{equation}
  \prox(x) = \max(\prox_{\tau |\cdot|_1}(x), 0)
\end{equation}


\section{Exercise 4}
Compute $\prox_{f}(x)$.


% \clearpage
% \printnoidxglossaries

\end{document}