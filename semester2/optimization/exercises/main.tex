\documentclass{article}

\usepackage{amsmath}
\usepackage{amssymb}
\usepackage{graphicx}
\usepackage{hyperref}
\hypersetup{
  colorlinks=true,
  linkcolor=blue, 
}

\newcommand{\E}{\mathbb{E}}
\newcommand{\R}{\mathbb{R}}
\newcommand{\N}{\mathbb{N}}
\newcommand{\1}{\mathbf{1}}
\renewcommand{\P}{\mathbb{P}}
\renewcommand{\L}{\mathcal{L}}
\newcommand{\ie}{\textit{i.e. }}
\newcommand{\eg}{\textit{e.g. }}

\title{Optimization - Minitest 3}
\author{Joris LIMONIER}
\date{February 1, 2022}
\begin{document}
\maketitle

\section*{Question 1}
\begin{equation}
  \label{lemma}
  \E_{\xi_k}\left[F(w_{k+1})\right] - F(w_k) \leq \alpha_k \left(\mu - \frac{1}{2} \alpha_k L M_G\right) \left\| \nabla F(w_k)\right\|^2 + \frac{1}{2} \alpha_k^2 LM
\end{equation}
\paragraph{What does equation \eqref{lemma} imply ?}
If \(\alpha_k\) is small enough, \(\E\left[F(w_{k+1})\right] \leq F(w_k)\), so we can hope that the gradient method leads to the minimum of \(F\).

\section*{Question 2}
\paragraph{What is the role of each of the components (\(\alpha_k, L, M, \ldots\))?}


\end{document}