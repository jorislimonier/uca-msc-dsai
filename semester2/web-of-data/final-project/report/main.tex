\documentclass[12pt]{article}
\usepackage[utf8]{inputenc}

% Importing settings from setup.sty
\usepackage{setup}
\usepackage{booktabs}
\usepackage{multicol}
\usepackage{multirow}
\usepackage{glossaries}
% \makenoidxglossaries
% \newcommand{\thetahat}{\hat{\theta}}

\newacronym{emr}{EMR}{Electronic Medical Record}
\newacronym{bmi}{BMI}{Body Mass Index}


% \pagenumbering{roman}
\begin{document}

% Inserting title page
\import{./}{title}

\pagenumbering{gobble}
\tableofcontents
% \listoffigures
% \listoftables



\newgeometry{
    left=25mm,
    right=25mm,
    top=25mm,
    bottom=25mm}
\pagenumbering{arabic}

\section{Model presentation}
The goal of this project is to model an \glspl{emr} using the XML syntax. We use a mix of foaf, schema and custom namespaces. The structure revolves around an \(EMR\) object, which contains informations including:
\begin{itemize}
    \item \(ns:belongsTo\)
    \item \(foaf:name\)
    \item \(foaf:age\)
    \item \(schema:weight\)
    \item \(schema:height\)
    \item \(ns:hasAllergy\)
    \item \(ns:reimbursement\)
    \item \(ns:surgery\)
    \item \(ns:consultation\)
\end{itemize}
where the \(ns:belongsTo\) predicate indicates the owner of the \gls{emr}. The other predicates give medical or personal information and should be self-explanatory. \\
The \(ns:consultation\) predicate has range \(ns:Consultation\), which is the generic class for consultation instances. Consultations then hold pieces of information about a consultation. That includes:
\begin{itemize}
    \item \(ns:prescription\)
    \item \(ns:hasPhysician\)
    \item \(ns:diagnosis\)
    \item \(ns:price\)
    \item \(ns:date\)
\end{itemize}


\section{Queries}
In this section, we present a number of queries to study our RDF and its schema
All queries in this section should be performed with the following prefices defined:
\begin{verbatim}
prefix foaf: <http://xmlns.com/foaf/0.1/>
prefix schema: <https://schema.org/>
prefix ns: <http://www.erm.fr/2022/01/01/ns.rdfs#>
prefix inst: <http://www.erm.fr/2022/01/01/inst.rdfs#>
\end{verbatim}

\subsection{Show all triples in the namespace \textit{ns}}
We know that a triple is composed of Subject, Predicate and Object. This query allows to see all defined triples, which have an object in the \(ns\) namespace.
\begin{verbatim}
select * where { ?subject a ?object . filter(strstarts(?object, ns:)) }
\end{verbatim}

\subsection{Find people whose age is even}
Let \(\lfloor x \rceil\) be defined as the rounding of \(x\) to the nearest integer. If \(x\) is even, then we get:
\begin{equation}
    \label{eqn: isEven}
    \frac{\lfloor x \rceil}{2} = \left \lfloor \frac{x}{2} \right \rceil
\end{equation}
we check equation \eqref{eqn: isEven} with the following SPARQL query:
\begin{verbatim}
select ?person ?ageEven
    where{
    ?emr ns:belongsTo ?person
    ?emr foaf:age ?age .
    bind (xsd:integer(?age/2) = xsd:integer(?age)/2 as ?ageEven)
}
\end{verbatim}

\subsection{Find people with age bigger than 24 or age less than 15}
\begin{verbatim}
select * where {
    ?emr a ns:EMR
{ ?emr foaf:age ?age .
filter (?age > 24) }
union
{ ?emr foaf:age ?age .
filter (?age < 15) }
}
\end{verbatim}


\subsection{Construct graph with youngerThan relationship}
Construct a graph of People, with relationships \(youngerThan\) if their age is less than the person they are being compared to.
\begin{verbatim}
construct {?person1 h:youngerThan ?person2}
where {
    ?emr1 ns:belongsTo ?person1 .
    ?emr2 ns:belongsTo ?person2 .
    ?emr1 foaf:age ?age1 
    ?emr2 foaf:age ?age2 
    filter (?age1 < ?age2)
}
\end{verbatim}

\subsection{Use OWL to infer \textit{Person}'s}
Get all people in the data. Let ``\(\subseteq\)'' denotes ``is a subclass of''. The OWL syntax allows us to deduce the following:
\begin{equation}
    ns:Infectiologist \subseteq ns:Physician \subseteq foaf:Person
\end{equation}
Thus the following query also returns \(ns:Raoult\), which is a \(ns:Infectiologist\).
\begin{verbatim}
select * where {
    ?person a foaf:Person
}
\end{verbatim}

\subsection{Link people to their physicians}
Get a list of pairs with people and physicians they had at least one consultation with.
\begin{verbatim}
select ?person ?physician where {
    ?emr ns:belongsTo ?person .
    ?emr a ns:EMR .
    ?emr ns:consultation ?consultation .
    ?consultation ns:hasPhysician ?physician
}
\end{verbatim}

\subsection{Find minors who took aspirin}
Let's say some medication (\textit{e.g.} aspirin) is found to be dangerous for minors (people under the age of 18). In this case, we would be interested in getting a list of people who took aspirin, then filter only those who are less than 18 years old.
\begin{verbatim}
select * where {
    ?emr foaf:age ?age
    ?emr ns:consultation ?consultation .
    ?consultation ns:prescription ?prescription .
    ?prescription ns:medication inst:Aspirin .
    filter(?age <= 18)
}
\end{verbatim}
  
\subsection{Compute BMI}
The \gls{bmi} is given by:
\begin{equation}
    \label{eqn: bmi m2}
    BMI = \frac{weight\ (kg^2)}{height^2\ (m^2)}
\end{equation}
but since our height is given in centimeters, equation \eqref{eqn: bmi m2} becomes:
\begin{equation}
    \label{eqn: bmi cm2}
    BMI = 10^4 \times \frac{weight\ (kg^2)}{height^2\ (cm^2)}
\end{equation}
which we compute using the following SPARQL query:
\begin{verbatim}
select ?person ?bmi where {
    ?emr ns:belongsTo ?person
    ?emr schema:height ?height
    ?emr schema:weight ?weight
    bind (10000*?weight/(?height*?height) as ?bmi)
}
\end{verbatim}
        
\subsection{Compute \textit{isObese}}
Starting from the previous example (\gls{bmi}), determine whether someone is obese. Note that we computed the condition for obesity based on the \(bmi > 22\), but this is only for demonstration purposes, since the actual criterion for obesity is \(bmi > 25\).
\begin{verbatim}   
insert { ?person ns:isObese ?obese } where {
    ?emr ns:belongsTo ?person
    ?emr schema:height ?height
    ?emr schema:weight ?weight
    bind (10000*?weight/(?height*?height) as ?bmi)
    bind (?bmi > 22 as ?obese)
}
\end{verbatim}
  
% \clearpage
% \printnoidxglossaries

\end{document}